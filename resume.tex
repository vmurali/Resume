%______________________________________________________________________________________________________________________
% @brief    LaTeX2e Resume for Kamil K Wojcicki
\documentclass[margin,line]{resume}


%______________________________________________________________________________________________________________________
\begin{document}
\name{\large Muralidaran Vijayaraghavan}
\begin{resume}

    %__________________________________________________________________________________________________________________
    % Contact Information
    \section{\mysidestyle Contact\\Information}

    CSAIL, MIT, Cambridge MA   \hfill Mobile: +1 408 839 3356          \\
    Homepage: people.csail.mit.edu/vmurali     \hfill Email: vmurali@csail.mit.edu    \\

    \vspace{-7mm}
    %__________________________________________________________________________________________________________________
    % Research Interests
    \section{\mysidestyle Research\\Interests}

    Automated theorem proving, formal verification and high-level languages for hardware design,\\
    correct-by-construction approach to hardware design, Computer architecture

    \vspace{-3mm}
    %__________________________________________________________________________________________________________________
    % Education
    \section{\mysidestyle Education}

    \textbf{Massachusetts Institute of Technology}, Cambridge, MA \\
    \vspace{-2mm}
    Ph.D., Electrical Engineering and Computer Science, \hfill \textbf{\textit{Expected: Fall 2014}}\\
    \vspace{-2mm}
    \begin{list2}
        \item Thesis topic: \textit{Specification and Formal Verification of Cache Coherence Protocols and Memory Models}
        \item Advisor:  Prof. Arvind
				\item The advent of highly concurrent, large scale distributed systems,
dramatically increases the impact of low-probability concurrency bugs. Because
of this, it is increasingly important that real systems are formally verified.
Current formal verification approaches can be divided into two categories: ones
which are highly automated but requires state space exploration limiting its
scalability (model checking), and those which are fully scalable but require
significant human intervention to provide deep insights and guide the tool
towards a complete proof (theorem proving).
This thesis lowers the human effort of theorem proving for distributed systems.
The verification of a system is factored into a generic reusable library
conveying the key insights of the domain, and the implementation-specific details.
This effectively provides a domain-specific reusable verification system which
can naturally fit into a designer's tool flow.
This is shown concretely in the context of cache coherence protocols, memory
models, and their implementations.

% There are two components in this thesis. The first component
%involves using Coq to model realistic directory-based hierarchical invalidation
%cache coherence protocols, and realistic speculating out-of-order processors,
%verifying each component separately and composing both the design and proofs of
%these individual components. The second part involves establishing an
%equivalence between operational semantics of parameterized hardware with
%axiomatic semantics of memory models. This establishes a bijective relation: the correct
%specification that system designers should use in order to implement a
%particular axiomatic memory model semantics, and the exact axiomatic semantics
%that a particular system would obey, enabling programmers to reason about their programs.
    \end{list2}\vspace{-3mm}
    S.M., Electrical Engineering and Computer Science, \hfill \textbf{\textit{February 2009}}\\
    \vspace{-3mm}
    \begin{list2}
        \item Thesis topic: \textit{Theory of composable latency-insensitive refinements}
        \item Advisor:  Prof. Arvind
    \end{list2}\vspace{-3mm}
    \textbf{Indian Institute of Technology, Madras}, Chennai, India\\
    B.Tech., Computer Science and Engineering, \hfill \textbf{\textit{June 2006}}\\
    \vspace{-7mm}


    %__________________________________________________________________________________________________________________
    % Professional Experience
    \section{\mysidestyle Work\\Experience}

    \textbf{Research Intern}, IBM T.J. Watson Research Center, Yorktown Heights, NY\\ 
    Supervisor: Dr. Kattamuri Ekanadham \\
    Developed a high-level language for writing synchronous hardware modules,
    and a tool to convert them automatically to latency-tolerant modules.\hfill \textbf{\textit{Summer 2010}}\\

    \vspace{-7mm}
    \textbf{Research Intern}, VSSAD group, Intel Corporation, Hudson, MA\\
    Supervisor: Prof. Joel Emer \\
    Helped develop \emph{HASim}, for building FGPA-based performance
          models for faster simulations and performance studies. Built an out-of-order superscalar
          processors based on MIPS R10000 using this infrastructure. \hfill \textbf{\textit{Summer 2007, Summer 2008}}\\
    \vspace{-9mm}
    %__________________________________________________________________________________________________________________
    % Publications
    \section{\mysidestyle Selected Publications}
    \begin{enumerate}
    \item \textbf{Vijayaraghavan, M.}, Dave, N., Arvind.
    ``Distributed Modular Hardware Compilation of Guarded Atomic Actions'' \textit{MEMOCODE 2013}
    \item Khan, A., \textbf{Vijayaraghavan, M.}, Boyd-Wickizer, S., Arvind.``Fast and cycle-accurate modeling of 
    a multicore processor''
    \textit {ISPASS 2012}
    \item Khan, A., \textbf{Vijayaraghavan, M.}, Arvind. ``A general technique for deterministic model-cycle-level debugging''
    \textit{MEMOCODE 2012}
    %\item Pellauer, M., Agarwal, A., Khan, A., Ng, M. C., \textbf{Vijayaraghavan, M.}, Brewer, F., Emer, J. S. 
    %``Design contest overview: Combined architecture for network stream categorization and intrusion detection (CANSCID)''
    %\textit{MEMOCODE 2010}
    \item Pellauer, M., \textbf{Vijayaraghavan, M.}, Adler, M., Arvind, Emer, J. S.
    ``A-Port Networks: Preserving the Timed Behavior of Synchronous Systems for
    Modeling on FPGAs'' \textit{TRETS (2009)}
    %\item Agarwal, A., Dave, N., Fleming, K., Khan, A., King, M., Ng, M. C., \textbf{Vijayaraghavan M.} ``
    %Implementing a fast cartesian-polar matrix interpolator'' \textit{MEMOCODE 2009}
    \item \textbf{Vijayaraghavan, M.}, Arvind. ``Bounded Dataflow Networks and
    Latency-Insensitive circuits'' \textit{MEMOCODE 2009}
    \item Pellauer, M.,
    \textbf{Vijayaraghavan, M.}, Adler, M., Arvind, Emer, J. S. ``A-Ports: an
    efficient abstraction for cycle-accurate performance models on FPGAs''
    \textit{FPGA 2008}
    \item Pellauer, M., \textbf{Vijayaraghavan, M.}, Adler, M., Arvind, Emer,
    J. S. ``Quick Performance Models Quickly: Closely-Coupled Partitioned
    Simulation on FPGAs'' \textit{ISPASS 2008}
    %\item Fleming, K., King, M., Ng, M. C., Khan, A., \textbf{Vijayaraghavan, M.} ``High-throughput Pipelined Mergesort'' \textit{MEMOCODE 2008}
    \item Ng, M. C., \textbf{Vijayaraghavan, M.}, Dave, N., Arvind, Raghavan, G., Hicks, J. ``From WiFi to WiMAX: Techniques for High-Level IP Reuse across Different OFDM Protocols'' \textit{MEMOCODE 2007}
    %\item Dave, N., Fleming, K., King, M., Pellauer, M., \textbf{Vijayaraghavan, M.} ``Hardware Acceleration of Matrix Multiplication on a Xilinx FPGA'' \textit{MEMOCODE 2007}
    \end{enumerate}

    %__________________________________________________________________________________________________________________
    % Honours and Awards
    %\section{\mysidestyle Honours and\\Awards} 

    %All India Rank 15 in the Joint Entrance Examination for admission into IITs, June 2002 \\
    %Highest GPA in the Department of Computer Science and Engineering at IIT Madras, June 2003\\


    %__________________________________________________________________________________________________________________
    % Computer Skills
    %\section{\mysidestyle Programming Language expertise} 

    %C, C++, Haskell, Coq, Bluespec
\end{resume}
\end{document}


%______________________________________________________________________________________________________________________
% EOF

